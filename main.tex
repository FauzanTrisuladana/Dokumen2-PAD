\documentclass[a4paper,oneside,11pt]{book}

% Package dasar
\usepackage[utf8]{inputenc}     % untuk karakter UTF-8
\usepackage[T1]{fontenc}        % encoding font
\usepackage{graphicx}
\usepackage{titling}
\usepackage{titlesec}
\usepackage[backend=biber,style=ieee]{biblatex}
\addbibresource{ref.bib}

% Package tambahan untuk layout dan formatting
\usepackage{geometry}           % untuk margin
\usepackage{setspace}           % untuk spasi baris
\usepackage{fancyhdr}           % untuk header/footer
\usepackage{hyperref}           % untuk link dan bookmark
\usepackage{listings}           % untuk code Python
\usepackage{xcolor}             % untuk warna
\usepackage{float}              % untuk posisi gambar
\usepackage{tocloft}            % Untuk membuat daftar isi menampilkan "Bab 1. Pendahuluan" instead of "1. Pendahuluan",
\usepackage{lipsum}             % untuk lorem ipsum
\usepackage{indentfirst}        % untuk indent paragraf pertama

% Pengaturan layout
\geometry{left=3cm, right=2cm, top=2cm, bottom=2cm}
\onehalfspacing  % spasi 1.5

% Format chapter
\titleformat{\chapter}[hang]
  {\normalfont\huge\bfseries\centering}{\chaptertitlename\ \thechapter.}{1em}{}

% Format code Python
\lstset{
    language=PHP,
    basicstyle=\ttfamily\small,
    keywordstyle=\color{blue}\bfseries,
    commentstyle=\color{gray},
    stringstyle=\color{red},
    numbers=left,
    numberstyle=\tiny,
    frame=single,
    breaklines=true,
    showstringspaces=false,
    xleftmargin=2cm,
    xrightmargin=1cm,
    morekeywords={Route,Controller,Request,Response,Blade,Model,Migration,Artisan,middleware,Auth,extends,section,yield,csrf,foreach,endif}
}

% Pengaturan hyperref
\hypersetup{
    colorlinks=true,
    linkcolor=black,
    filecolor=magenta,
    urlcolor=blue,
    citecolor=red
}

% Info dokumen
\title{Dokumen \\ Proyek Aplikasi Dasar\\
UTS}
\author{Kelompok 21 (SIA WEB)\\
Anggota: \\
1. Nama 1 (123456789)\\
2. Nama 2 (123456789)\\
3. Nama 3 (123456789)\\
4. Nama 4 (123456789)}

\begin{document}

% Title Page
\begin{titlingpage} 
\begin{center}

\vspace{4cm} 
\begin{Large} 
\textbf{\thetitle} \\
\end{Large}
\vspace{2cm}

\includegraphics[height=8cm]{lambang ugm.png}\\ 
\begin{large}
\vspace{2cm} 
\theauthor\\ 
\end{large}

\vspace{2cm}
\begin{Large}
\textbf{Sekolah Vokasi}\\
\textbf{Universitas Gadjah Mada}\\
\textbf{Yogyakarta}\\
\textbf{2025}\\
\end{Large}

\end{center}
\end{titlingpage}

% Header/Footer
\pagestyle{fancy}
\fancyhf{}            % kosongkan header/footer default
\fancyfoot[C]{\thepage} % taruh nomor halaman di footer tengah


% --- Styling Judul Daftar Isi, Gambar, dan Kode ---
\renewcommand{\cfttoctitlefont}{\hfill\normalfont\LARGE\bfseries}
\renewcommand{\cftaftertoctitle}{\hfill}
\renewcommand{\cftbeforetoctitleskip}{1em}
\renewcommand{\cftaftertoctitleskip}{1.5em}

% --- Format Bab di Daftar Isi ---
\renewcommand{\cftchappresnum}{Bab~}
\renewcommand{\cftchapaftersnum}{.}
\renewcommand{\cftchapnumwidth}{3.5em}
\renewcommand{\cftchapfont}{\bfseries}
\renewcommand{\cftchappagefont}{\bfseries}

% --- Spasi dan Estetika ---
\setlength{\cftbeforechapskip}{0.8em} % jarak antar bab
\setlength{\cftbeforesecskip}{0.2em}  % jarak antar subbab
\renewcommand{\cftsecfont}{\normalfont}
\renewcommand{\cftsubsecfont}{\normalfont}

% --- Judul Lokal ---
\renewcommand{\contentsname}{\centering Daftar Isi}
\renewcommand{\listfigurename}{\centering Daftar Gambar}
\renewcommand{\lstlistlistingname}{\centering Daftar Kode}

% --- Gaya Daftar Gambar dan Daftar Kode ---
\renewcommand{\cftloftitlefont}{\hfill\normalfont\LARGE\bfseries}
\renewcommand{\cftafterloftitle}{\hfill}

\tableofcontents

\newpage
% Daftar Gambar
\listoffigures

\newpage

% Dokumen-Dokumen penting
\chapter{Dokumen}
\section{MVP}
\section{Bisreq}
\section{Use Case}
\section{Activity Diagram}
\section{ERD}
\section{Wireframe}

% Progress
\chapter{Progress}
\section{Slicing}
\section{API}
\section{Dokumentasi API}

\end{document}